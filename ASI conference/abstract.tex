\documentclass[12pt,a4paper]{article}
\usepackage[utf8]{inputenc}
\title{A machine learning approach to restoration of globular cluster images}
\author{Rahul Yedida, Bhavana KV, Snehanshu Saha, Margarita Safonova}
\date{}
\begin{document}
\maketitle
\begin{abstract}
Successful differential photometry of crowded fields requires images to be of a relatively uniform quality. Modern DIA (Differential Imaging Analysis) techniques usually take care of effects of variable atmospheric extinction and exposure times, and the methods even work better as the crowding increases, because in denser fields more pixels contain information about the PSF difference. However, if a time series is taken over long baselines, other sources of noise can creep in and worsen the images, such as e.g. PSF anisotropy, or smearing, which DIA usually cannot handle even for moderately elongated PSFs. This mostly happens due to either strong winds during the exposure or due to incorrect orientation of lenses in the image sensor device. This causes stars to be out-of-shape and/or out-of-focus, thereby making it improper to use in the subtraction process of DIA because these “corrupt” images lead to false positive results of variability/transients. In most of the studies that use DIA, these images are removed from the data set. However, when the observing program is over a long baseline and the exposures are taken at a low cadence, each image becomes very important as it is not possible to repeat them, and each image becomes essential. We investigate the efficacy of employing machine learning approaches to the problem of correcting such images to restore the circular shapes of stars as a first step, in order to minimize data loss in the currently run programs based on optical time series.
\end{abstract}
\textbf{Keywords:} machine learning; deep learning; neural networks; image restoration
\end{document}